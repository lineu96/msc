\begin{resumo}

Ciência de dados é um campo de estudo interdisciplinar que compreende áreas como estatística, ciência da computação e matemática. Neste contexto, métodos estatísticos são de fundamental importância sendo que, dentre as possíveis técnicas disponíveis para análise de dados, os modelos de regressão têm papel importante. Tais modelos são indicados a problemas nos quais existe interesse em verificar a associação entre uma ou mais variáveis respostas e um conjunto de variáveis explicativas. Isto é feito através da obtenção de uma equação que explique a relação entre as variáveis explicativas e a(s) resposta(s). Existem modelos uni e multivariados: nos modelos univariados há apenas uma variável resposta; já em modelos multivariados há mais de uma resposta. Dentre as classes de modelos multivariados estão os modelos multivariados de covariância linear generalizada (McGLMs). No contexto de modelos de regressão, é comum o interesse em avaliar os valores dos parâmetros por meio de testes de hipóteses e existem técnicas baseadas em tais testes, como as análises de variâncias univariadas, multivariadas e ainda os testes de comparações múltiplas. No entanto, considerando os McGLMs, não há discussão a respeito do uso destes testes para a classe. Nossa proposta é utilizar o teste Wald para a realização de testes de hipóteses gerais sobre parâmetros de regressão e dispersão de McGLMs. Por meio da avaliação dos parâmetros de regressão é possível identificar qual(is) variável(is) explicativa(s) apresentam efeito sobre a(s) resposta(s). Por meio do estudo dos parâmetros de dispersão pode-se avaliar o efeito da correlação entre unidades do estudo, como por exemplo em estudos longitudinais, temporais e de medidas repetidas. Apresentamos implementações em R de funções para efetuar tais testes, bem como funções para efetuar ANOVAs, MANOVAs e testes de comparações múltiplas. As propriedades e comportamento dos testes propostos foram verificados com base em estudos de simulação e o potencial de aplicação das metodologias discutidas foi motivado com base na aplicação a um conjunto de dados real. Os resultados mostraram que quanto mais distante a hipótese testada é dos valores verdadeiros dos parâmetros, maior é o percentual de rejeição da hipótese nula. Tal como esperado, os menores percentuais de rejeição foram observados quando a hipótese nula testada correspondia aos reais valores dos parâmetros. Também verificou-se que conforme aumenta-se o tamanho amostral, o percentual de rejeição aumenta para hipóteses nulas pouco diferentes dos valores simulados dos parâmetros. Logo, os resultados apontam que o teste Wald pode ser usado para avaliar hipóteses sobre parâmetros de regressão e dispersão de McGLMs, o que permite uma melhor interpretação do efeito das variáveis e estruturas de delineamento em contextos práticos.

\end{resumo}
