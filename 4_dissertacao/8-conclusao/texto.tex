
\chapter{Considerações finais}

%=====================================================

\section{Conclusões gerais}

\begin{itemize}
  \item Resultados do estudo de simulação
  \item Discussão bernoulli (o teste nao funcionou pq a estimação é comprometida)
  \item Discussão redução do poder conforme aumenta o numero de parametros testados.
  \item Ressaltar que a proposta é válida para todas as referências que são submodelos do McGLM.
\end{itemize}

%=====================================================

\section{Limitações}

\begin{itemize}
  \item O que acontece quando combina-se parâmetros de diferentes tipos no mesmo teste.
  \item O que acontece em cenários longitudinais desbalanceados.
  \item O que acontece quando o numero de parametros nos testes aumenta.
\end{itemize}

%=====================================================

\section{Trabalhos futuros}

\begin{itemize}
  \item Trabalhar em outros testes (LRT e LMT)
  \item Implementar novos procedimentos para comparações múltiplas
  \item Adequação a diferentes contrastes
  \item Explorar procedimentos para seleção automática de covariáveis
  \item Explorar seleção de covariáveis por meio de inclusão de penalização no ajuste por complexidade. Similar a ideia de regressão por splines.
\end{itemize}

Seleção automática:

Backward elimination:
  - Comece com o modelo completo.
  - Compare esse modelo com todos os modelos ajustados sem uma das covariáveis e calcule o p-valor.
  - Elimine o termo com maior p-valor que seja maior que algum limiar.
  - Repita até que nenhuma covariável possa ser eliminada.

Forward selection
  - Comece com o modelo sem covariáveis.
  - Avalie a adição de cada covariável usando o p-valor.
  - Adicione a covariável que mais se destaca.
  - Repita até que nenhuma covariável estatísticamente significante reste.

Stepwise selection: Mistura o forward e o backward.

%=====================================================
