\chapter{Estudo de simulação}

%=====================================================

Com o objetivo de avaliar o poder do teste Wald em testes de hipóteses sobre parâmetros de McGLMs, foram executados estudos de simulação. Nestas simulações avaliamos o comportamento da proposta para três distribuições de probabilidade: Normal, Poisson e Bernoulli. Simulamos cenários univariados e também trivariados com diferentes tamanhos amostrais para verificar as propriedades dos testes sobre parâmetros de regressão, dispersão e potência.

Para simular conjuntos de dados univariados foram usadas bibliotecas padrões do R. Para simular conjuntos de dados com múltiplas respostas seguindo distribuição Normal, foi usada a biblioteca R \emph{mvtnorm} \citep{mvtnorm1}, \citep{mvtnorm2}. Para as outras distribuições foi utilizado o método NORTA \citep{cario1997modeling} implementado na biblioteca R \emph{NORTARA} \citep{nortara}.

\section{Parâmetros de regressão}

Para avaliação de hipóteses sobre parâmetros de regressão foram considerados tamanhos amostrais de 50, 100, 250, 500 e 1000. Foram gerados 500 conjuntos de dados para cada tamanho amostral simulando uma situação com uma variável explicativa categórica de 4 níveis. Para distribuição Normal os parâmetros de regressão usados foram: $\beta_0 = 5$, $\beta_1 = 0$, $\beta_2 = 0$, $\beta_3 = 0$. Para a distribuição Poisson os parâmetros de regressão usados foram: $\beta_0 = 2,3$, $\beta_1 = 0$, $\beta_2 = 0$, $\beta_3 = 0$. E para a distribuição Bernoulli os parâmetros de regressão usados foram: $\beta_0 = 0,5$, $\beta_1 = 0$, $\beta_2 = 0$, $\beta_3 = 0$. Os valores foram escolhidos de tal modo que o coeficiente de variação para distribuição Normal fosse de 20\%, as contagens para Poisson fossem próximas de 10 e a probabilidade de sucesso da Bernoulli fosse aproximadamente 0,6. Foram avaliados cenários univariados e trivariados com estas características. Para os cenários trivariados, existem 4 parâmetros por resposta que seguem as configurações descritas. Para cada amostra gerada foi ajustado um McGLM. As funções de ligação e variância para cada distribuição são apresentadas na \autoref{tab:link_var}. 

\begin{table}[H]
\centering
\begin{tabular}{ccc}
\hline
Distribuição & Função de ligação & Função de variância \\ \hline
Normal       & Identidade        & Constante           \\
Poisson      & Logarítmica       & Tweedie             \\
Bernoulli    & Logito            & Binomial            \\ \hline
\end{tabular}
\caption{Funções de ligação e variância utilizadas nos modelos para cada distribuição simulada.}
\label{tab:link_var}
\end{table}

Em todos os casos o preditor matricial para a matriz de variância e covariância foi especificado de forma a explicitar que as observações são independentes dentro de cada resposta. A correlação entre respostas no caso trivariado é dada pela matriz $\Sigma_b$ descrita na \autoref{eq:correlacao}.

\begin{equation} \label{eq:correlacao}
\Sigma_b = 
\begin{bmatrix}
1    & 0,75 & 0,5  \\
0,75 & 1    & 0,25 \\
0,5  & 0,25 & 1    \\
\end{bmatrix}
\end{equation}

Com os modelos ajustados, o procedimento consistiu em variar as hipóteses testadas sobre os parâmetros simulados. Consideramos 20 diferentes hipóteses baseadas em um decréscimo de $\beta_0/20$ e distribuição deste decréscimo nos demais $\beta$s da hipótese nula. Para cada ponto avaliamos o percentual de rejeição da hipótese nula. A ideia é verificar o que ocorre quando afastamos as hipóteses nulas dos reais valores dos parâmetros. Espera-se que no primeiro ponto haja um percentual de rejeição baixo, pois a hipótese nula corresponde aos reais valores dos parâmetros. Para os demais pontos espera-se que o percentual de rejeição aumente gradativamente, pois as hipóteses afastam-se dos valores originalmente simulados. As hipóteses testadas para as respostas com distribuição Normal são mostradas na \autoref{tab:th_normal}; as hipóteses testadas para respostas seguindo distribuição Poisson estão na \autoref{tab:th_poisson} e as hipóteses para as respostas Bernoullis são mostradas na \autoref{tab:th_bernoulli}.

\begin{table}[H]
\centering
\begin{tabular}{ccccc}
\hline
          & $\beta_0$ & $\beta_1$ & $\beta_2$ & $\beta_3$ \\ \hline
$H_{01}$  & 5         & 0         & 0         & 0         \\
$H_{02}$  & 4,75      & 0,083     & 0,083     & 0,083     \\
$H_{03}$  & 4,5       & 0,166     & 0,166     & 0,166     \\
$H_{04}$  & 4,25      & 0,25      & 0,25      & 0,25      \\
$H_{05}$  & 4         & 0,333     & 0,333     & 0,333     \\
$H_{06}$  & 3,75      & 0,416     & 0,416     & 0,416     \\
$H_{07}$  & 3,5       & 0,5       & 0,5       & 0,5       \\
$H_{08}$  & 3,25      & 0,583     & 0,583     & 0,583     \\
$H_{09}$  & 3         & 0,666     & 0,666     & 0,666     \\
$H_{010}$ & 2,75      & 0,75      & 0,75      & 0,75      \\
$H_{011}$ & 2,5       & 0,833     & 0,833     & 0,833     \\
$H_{012}$ & 2,25      & 0,916     & 0,916     & 0,916     \\
$H_{013}$ & 2         & 1         & 1         & 1         \\
$H_{014}$ & 1,75      & 1,083     & 1,083     & 1,083     \\
$H_{015}$ & 1,5       & 1,166     & 1,166     & 1,166     \\
$H_{016}$ & 1,25      & 1,25      & 1,25      & 1,25      \\
$H_{017}$ & 1         & 1,33      & 1,33      & 1,33      \\
$H_{018}$ & 0,75      & 1,41      & 1,41      & 1,41      \\
$H_{019}$ & 0,5       & 1,5       & 1,5       & 1,5       \\
$H_{020}$ & 0,25      & 1,58      & 1,58      & 1,58      \\ \hline
\end{tabular}
\caption{Hipóteses testadas para distribuição Normal.}
\label{tab:th_normal}
\end{table}

\begin{table}[H]
\centering
\begin{tabular}{ccccc}
\hline
          & $\beta_0$ & $\beta_1$ & $\beta_2$ & $\beta_3$ \\ \hline
$H_{01}$  & 2,3       & 0         & 0         & 0         \\
$H_{02}$  & 2,185     & 0,0383    & 0,0383    & 0,0383    \\
$H_{03}$  & 2,07      & 0,0766    & 0,0766    & 0,0766    \\
$H_{04}$  & 1,955     & 0,115     & 0,115     & 0,115     \\
$H_{05}$  & 1,84      & 0,153     & 0,153     & 0,153     \\
$H_{06}$  & 1,725     & 0,192     & 0,191     & 0,191     \\
$H_{07}$  & 1,61      & 0,23      & 0,23      & 0,23      \\
$H_{08}$  & 1,495     & 0,268     & 0,268     & 0,268     \\
$H_{09}$  & 1,38      & 0,306     & 0,306     & 0,306     \\
$H_{010}$ & 1,265     & 0,345     & 0,345     & 0,345     \\
$H_{011}$ & 1,15      & 0,383     & 0,383     & 0,383     \\
$H_{012}$ & 1,035     & 0,421     & 0,421     & 0,421     \\
$H_{013}$ & 0,92      & 0,46      & 0,46      & 0,46      \\
$H_{014}$ & 0,805     & 0,498     & 0,498     & 0,498     \\
$H_{015}$ & 0,69      & 0,536     & 0,536     & 0,536     \\
$H_{016}$ & 0,575     & 0,575     & 0,575     & 0,575     \\
$H_{017}$ & 0,46      & 0,613     & 0,613     & 0,613     \\
$H_{018}$ & 0,345     & 0,651     & 0,651     & 0,651     \\
$H_{019}$ & 0,23      & 0,69      & 0,69      & 0,69      \\
$H_{020}$ & 0,115     & 0,728     & 0,728     & 0,728     \\ \hline
\end{tabular}
\caption{Hipóteses testadas para distribuição Poisson.}
\label{tab:th_poisson}
\end{table}


\begin{table}[H]
\centering
\begin{tabular}{ccccc}
\hline
          & $\beta_0$ & $\beta_1$ & $\beta_2$ & $\beta_3$ \\ \hline
$H_{01}$  & 0,5       & 0         & 0         & 0         \\
$H_{02}$  & 0,475     & 0,008     & 0,008     & 0,008     \\
$H_{03}$  & 0,45      & 0,016     & 0,016     & 0,016     \\
$H_{04}$  & 0,425     & 0,025     & 0,025     & 0,025     \\
$H_{05}$  & 0,4       & 0,033     & 0,033     & 0,033     \\
$H_{06}$  & 0,375     & 0,041     & 0,041     & 0,041     \\
$H_{07}$  & 0,35      & 0,05      & 0,05      & 0,05      \\
$H_{08}$  & 0,325     & 0,058     & 0,058     & 0,058     \\
$H_{09}$  & 0,3       & 0,066     & 0,066     & 0,066     \\
$H_{010}$ & 0,275     & 0,075     & 0,075     & 0,075     \\
$H_{011}$ & 0,25      & 0,083     & 0,083     & 0,083     \\
$H_{012}$ & 0,225     & 0,091     & 0,091     & 0,091     \\
$H_{013}$ & 0,2       & 0,1       & 0,1       & 0,1       \\
$H_{014}$ & 0,175     & 0,108     & 0,108     & 0,108     \\
$H_{015}$ & 0,15      & 0,116     & 0,116     & 0,116     \\
$H_{016}$ & 0,125     & 0,125     & 0,125     & 0,125     \\
$H_{017}$ & 0,099     & 0,133     & 0,133     & 0,133     \\
$H_{018}$ & 0,074     & 0,141     & 0,141     & 0,141     \\
$H_{019}$ & 0,049     & 0,15      & 0,15      & 0,15      \\
$H_{020}$ & 0,024     & 0,158     & 0,158     & 0,158     \\ \hline
\end{tabular}
\caption{Hipóteses testadas para distribuição Bernoulli.}
\label{tab:th_bernoulli}
\end{table}

Para representar graficamente os resultados tomamos a distância euclideana de cada vetor de hipóteses com relação ao vetor usado para simular os dados. Adicionalmente dividimos o vetor de distâncias pelo seu desvio padrão para obter distâncias em uma mesma escala, independente dos parâmetros de regressão. Os resultados são apresentados na \autoref{}.


\section{Parâmetros de dispersão}

Para avaliação de hipóteses sobre parâmetros de dispersão foram considerados os mesmos tamanhos amostrais: 50, 100, 250, 500 e 1000. Contudo, os conjuntos de dados simulam uma situação em que cada unidade amostral fornece 5 medidas ao conjunto de dados. Foram gerados 500 conjuntos de dados para cada tamanho amostral e distribuição. Para distribuição Normal foram simulados vetores com média 5 e desvio padrão igual a 1. Para distribuição Poisson foram simuladas contagens com taxa igual a 10. Para distribuição Bernoulli foram simulados vetores de uma variável dicotômica com probabilidade de sucesso igual a 0,6.

Em todos os casos, os parâmetros de dispersão para gerar os conjuntos de dados foram fixados em 0,5 e não foi incluído efeito de variáveis explicativas. Foram avaliados cenários univariados e trivariados com estas características. Para cada amostra gerada foi ajustado um McGLM com funções de ligação e variância tal como descrito na \autoref{tab:link_var}. Nos cenários trivariados a correlação entre respostas é dada pela \autoref{eq:correlacao}.

Neste caso, como as medidas são correlacionadas dentro de cada resposta, é necessário especificar um preditor matricial que deixa explícito que as medidas são correlacionadas. O objetivo é testar hipóteses sobre os parâmetros de dispersão associados a este preditor matricial. 

Com os modelos ajustados, o procedimento consistiu em variar as hipóteses testadas sobre os parâmetros simulados. Consideramos 20 diferentes hipóteses baseadas em um decréscimo sucessivo de $0,5/20$ para cada hipótese nula testeda. Para cada ponto avaliamos o percentual de rejeição da hipótese nula. A ideia é afastar sucessivamente a hipótese dos valores simulados e avaliar se conforme afastamos a hipótese dos valores verdadeiros, o percentual de rejeição aumenta. As hipóteses testadas são mostradas na \autoref{tab:th_taus}.

\begin{table}[H]
\centering
\begin{tabular}{ccc}
\hline
          & $\tau_0$ & $\tau_1$ \\ \hline
$H_{01}$  & 0,5      & 0,5      \\
$H_{02}$  & 0,475    & 0,475    \\
$H_{03}$  & 0,45     & 0,45     \\
$H_{04}$  & 0,425    & 0,425    \\
$H_{05}$  & 0,4      & 0,4      \\
$H_{06}$  & 0,375    & 0,375    \\
$H_{07}$  & 0,35     & 0,35     \\
$H_{08}$  & 0,325    & 0,325    \\
$H_{09}$  & 0,3      & 0,3      \\
$H_{010}$ & 0,275    & 0,275    \\
$H_{011}$ & 0,25     & 0,25     \\
$H_{012}$ & 0,225    & 0,225    \\
$H_{013}$ & 0,2      & 0,2      \\
$H_{014}$ & 0,175    & 0,175    \\
$H_{015}$ & 0,15     & 0,15     \\
$H_{016}$ & 0,125    & 0,125    \\
$H_{017}$ & 0,099    & 0,099    \\
$H_{018}$ & 0,074    & 0,074    \\
$H_{019}$ & 0,049    & 0,049    \\
$H_{020}$ & 0,024    & 0,024    \\ \hline
\end{tabular}
\caption{Hipóteses testadas para parâmetros de dispersão.}
\label{tab:th_taus}
\end{table}

Do mesmo modo que foi feito para os parâmetros de regressão, foi tomada a distância euclideana de cada vetor de hipóteses com relação ao vetor usado para simular os dados; e o vetor de distâncias foi padronizado para obter distâncias em uma mesma escala, independente dos parâmetros simulados. Os resultados são apresentados na \autoref{}.

\section{Parâmetros de potência}

Para avaliação de hipóteses sobre parâmetros de potência foram novamente considerados tamanhos amostrais de 50, 100, 250, 500 e 1000. Foram gerados 500 conjuntos de dados para cada tamanho amostral simulando uma situação com uma variável explicativa categórica de 4 níveis. Para distribuição Normal os parâmetros de regressão usados foram: $\beta_0 = X$, $\beta_1 = X$, $\beta_2 = X$, $\beta_3 = X$. Para a distribuição Poisson os parâmetros de regressão usados foram: $\beta_0 = X$, $\beta_1 = X$, $\beta_2 = X$, $\beta_3 = X$. E para a distribuição Bernoulli os parâmetros de regressão usados foram: $\beta_0 = X$, $\beta_1 = X$, $\beta_2 = X$, $\beta_3 = X$. Os valores foram escolhidos de tal modo que haja ao menos um parâmetro de regressão com efeito significativo para que seja possível estimar os parâmetros de potência. Foram avaliados cenários univariados e trivariados com estas características.

Para cada amostra gerada foi ajustado um McGLM com funções de ligação e variância descritas na \autoref{tab:link_var}. A correlação entre respostas no caso trivariado é dada pela matriz $\Sigma_b$ descrita na \autoref{eq:correlacao}. Em todos os casos o preditor matricial foi especificado de forma a explicitar que as observações são independentes dentro de cada resposta.

Similar ao que foi feito para parâmetros de regressão e dispersão, o procedimento consistiu em variar as hipóteses testadas sobre os parâmetros simulados. 

Nestas configurações, espera-se que o parâmetro de potência seja X,Y e Z para as distribuições Normal, Poisson e Bernoulli, respectivamente .

Para avaliar o teste, consideramos 20 diferentes hipóteses baseadas em um decréscimo sucessivo de $p/20$ para cada hipótese nula testeda. Para cada ponto avaliamos o percentual de rejeição da hipótese nula. 

Tal como nos casos anteriores espera-se que, afastando os valores da hipótese dos valores verdadeiros, o número de rejeições cresça.

As hipóteses testadas são mostradas na \autoref{tab:th_p}.


Da mesma forma que o realizado para o estudo sobre parâmetros de regressão e dispersão,  tomamos a distância euclideana de cada vetor de hipóteses com relação ao vetor usado para simular os dados e padronizamos o vetor. Os resultados são apresentados na \autoref{}.

%=====================================================

\textbf{TODO}

\begin{itemize}
  
  \item \textbf{ACRESCENTAR GRÁFICOS}
  
  \item \textbf{TERMINAR SUBSEÇÃO PARA PARAMETRO DE POTENCIA}
  
\end{itemize}


%=====================================================
