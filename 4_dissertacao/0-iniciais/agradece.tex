\begin{agradece}	% só gera conteúdo se for na versão final

À minha família, em especial a meus pais, Hamilton Alves de Freitas e Lúcia Elena Cavazani de Freitas, que mesmo atuando distante de sua realidade nunca deixaram de estar ao meu lado.

Ao meu orientador, Professor Doutor Wagner Hugo Bonat, que me acompanha nesta jornada desde muito antes do ingresso na pós graduação.

Ao professor Professor Doutor Walmes Marques Zeviani, apoiador e conselheiro deste e outros projetos desde a graduação.

Aos professores Wagner e Walmes também agradeço pela confiança depositada em todos os projetos que participei na Ômega - Escola de Data Science.

Aos professores Cesar Augusto Taconeli e José Luiz Padilha da Silva, meus primeiros orientadores em minha trajetória acadêmica e grandes apoiadores da sequência da minha carreira com foco em pesquisa e docência.

Ao professor doutor Paulo Justiniano Ribeiro Junior, pelas experiências ao seu lado na disciplina transversal de métodos estatísticos em pesquisa científica na Universidade Federal do Paraná - MEPC, pelas oportunidades, conversas, risadas e pela maneira gentil que sempre me tratou.

Ao professor doutor Marco Antonio Zanata Alves pelo acolhimento, paciência, por confiar no meu potencial para participar de projetos de linhas de pesquisa que não estou habituado, e pelos valiosos conselhos sobre como tratar do tema desta dissertação para com pessoas com formações diferentes das minhas.

Aos pesquisadores Ligia Oliveira Carlos, Marilia Ramos, Nathalia Wagner, Ingrid Felicidade e Antonio Carlos Campos, do grupo de pesquisa em obesidade, cirurgia bariátrica e microbioma do departamento de clínica cirurgica da Universidade Federal do Paraná, pela disponibilização do conjunto de dados usado para motivar as ideias deste trabalho; em especial à Doutora Ligia Carlos pelo suporte prestado ao longo da elaboração e revisão do trabalho.

A todos os colegas de mestrado, em especial àqueles do grupo de pesquisa HiPES, pelo acolhimento e paciência com um novo colega de formação diferente.

Aos professores dos departamentos de Estatística, Informática e Matemática que fizeram parte da minha trajetória na Universidade Federal do Paraná. Em especial àqueles do Laboratório de Estatística e Geoinformação. 

À Universidade Federal do Paraná - UFPR e ao Programa de Pós Graduação em Informática - PPGINF, incluindo professores e equipe administrativa, pelo brilhante trabalho e resiliência apresentados mesmo em tempos de pandemia.

À Coordenação de Aperfeiçoamento Pessoal de Ensino Superior - CAPES, pelo suporte financeiro nestes anos de pesquisa.

Aos membros da banca de qualificação e defesa da dissertação pelas valiosas contribuições.

À todos que estiveram ao meu lado no decorrer destes anos e contribuíram direta ou indiretamente neste trabalho. 

À todos vocês, meu sincero agradecimento.

\end{agradece}
