\begin{abstract}

Data science is an interdisciplinary field of study that comprises areas such as statistics, computer science and mathematics. In this context, statistical methods are of fundamental importance and, among the possible techniques available for data analysis, regression models play an important role. Such models are suitable for problems in which there is an interest in verifying the association between one or more response variables and a set of explanatory variables. This is done by obtaining an equation that explains the relationship between the explanatory variables and the response(s). There are univariate and multivariate models: in univariate models there is only one response variable; in multivariate models there is more than one response. Among the classes of multivariate models are the multivariate covariance generalized linear models (McGLMs). In the context of regression models, there is a common interest in evaluating parameter values through hypothesis tests and there are techniques based on such tests, such as univariate and multivariate analyzes of variance and even multiple comparison tests. However, considering the McGLMs, there is no discussion regarding the use of these tests for the class. Our proposal is to use the Wald test to carry out tests of general hypotheses on regression and dispersion parameters of McGLMs. By evaluating the regression parameters, it is possible to identify which explanatory variable(s) have an effect on the response(s). Through the study of dispersion parameters, the effect of the correlation between study units can be evaluated, for example in longitudinal, temporal and repeated measures studies. We present R implementations of functions to perform such tests, as well as functions to perform ANOVAs, MANOVAs and multiple comparison tests. The properties and behavior of the proposed tests were verified based on simulation studies and the potential of application of the discussed methodologies was motivated based on the application to a real dataset. The results showed that the further the tested hypothesis is from the true values of the parameters, the greater the percentage of rejection of the null hypothesis. As expected, the lowest rejection percentages were observed when the null hypothesis tested corresponded to the real values of the parameters. It was also verified that as the sample size increases, the rejection percentage increases for null hypotheses that are little different from the simulated values of the parameters. Therefore, the results indicate that the Wald test can be used to evaluate hypotheses about regression and dispersion parameters of McGLMs, which allows a better interpretation of the effect of variables and design structures in practical contexts.

\end{abstract}
