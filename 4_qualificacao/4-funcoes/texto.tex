
\chapter{Funções implementadas}

\label{cap:funcoes}

% figuras estão no subdiretório "figuras/" dentro deste capítulo
%\graphicspath{\currfiledir/figuras/}

No capítulo anterior vimos que podemos chegar a um teste de hipóteses sobre qualquer um dos parâmetros de uma classe de modelos em que podemos especificar modelos para lidar com múltiplas respostas, de diferentes naturezas, modelando também a correlação entre indivíduos da amostra, os McGLM. Deste modo um dos objetivos deste trabalho consiste em implementar tais testes no software R \citep{softwareR} com o objetivo de complementar as já possíveis análises permitidas pelo pacote \emph{mcglm} \citep{mcglm}.

Baseando-nos nas funções do pacote \emph{car} \citep{car}, implementamos funções temos funções para efetuar Análises de Variância por variável resposta (ANOVA); Análises de Variância multivariadas (MANOVA), note que no caso da MANOVA os preditores devem ser iguais para todas as respostas sob análise; Análise de variância focada no preditor linear matricial, em que o objetivo é verificar a significância dos parâmetros de dispersão; e uma função para hipóteses lineares gerais em que todos os elementos são especificadas pelo usuário, na qual é possível testar hipóteses sobre parâmetros de regressão, dispersão ou potência. Todas as funções implementadas podem ser acessadas em \href{https://github.com/lineu96/msc/blob/master/3_th_mcglm/0_funcoes/functions.R}. A \autoref{tab:funcoes} mostra os nomes e descrições das funções implementadas.

\begin{table}[h]
\centering
\begin{tabular}{ll}
\hline
Função                   & Descrição                                            \\ \hline
mc\_anova\_I()           & ANOVA  tipo I   (imita uma sequencial)               \\
mc\_anova\_II()          & ANOVA  tipo II  (nao bate com o car)                 \\
mc\_anova\_III()         & ANOVA  tipo III                                      \\
mc\_anova\_disp()        & ANOVA  tipo III para dispersão                       \\
mc\_manova\_I()          & MANOVA tipo I   (imita uma sequencial)               \\
mc\_manova\_II()         & MANOVA tipo II  (nao bate com o car)                 \\
mc\_manova\_III()        & MANOVA tipo III                                      \\
mc\_manova\_disp()       & MANOVA tipo III para dispersão                       \\
mc\_linear\_hypothesis() & Hipóteses lineares gerais especificadas pelo usuário \\ \hline
\end{tabular}
\caption{Funções implementadas}
\label{tab:funcoes}
\end{table}

\section{mc\_anova\_I()}

\section{mc\_anova\_II()}

\section{mc\_anova\_III()}

\section{mc\_anova\_disp()}

\section{mc\_manova\_I()}

\section{mc\_manova\_II()}

\section{mc\_manova\_III()}

\section{mc\_manova\_disp()}

\section{mc\_linear\_hypothesis()}

